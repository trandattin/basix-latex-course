\documentclass[12pt, a4paper]{article}
\usepackage[utf8]{vietnam}
\usepackage[left=3cm, right=2cm, top=2cm, bottom=2cm]{geometry}
\usepackage{indentfirst}
\setlength{\parindent}{1cm}
\usepackage{fancyhdr}
\fancyhf{}
\pagestyle{fancy}
\rfoot{trang \thepage}
\lhead{LCĐ-LCH VIỆN TOÁN ỨNG DỤNG VÀ TIN HỌC}
\rhead{\textbf{\textit{\LaTeX \ Basics \\ Định dạng văn bản}}}
\renewcommand{\headrulewidth}{2pt}
\renewcommand{\footrulewidth}{1pt}
\usepackage{watermark}
\usepackage{draftwatermark}
\SetWatermarkText{\LaTeX \ Basics}
\SetWatermarkScale{0.6}

\begin{document}

\textbf {\LARGE HỆ MẶT TRỜI} 
 
 \textbf{Hệ Mặt Trời} (hay \textbf{Thái Dương Hệ}) là  1 \textit{hệ hành tinh có Mặt Trời} ở trung tâm và \textit{các thiên thể} nằm trong phạm vi lực hấp dẫn của Mặt Trời, tất cả chúng được hình thành từ sự suy sụp của một\textit{ đám mây phân tử } khổng lồ cách đây 4,6 tỉ năm. Đa phần các thiên thể quayquanh Mặt Trời, và khối lượng tập trung chủ yếu vào 8 hành tinh có quỹ đạo gần tròn và mặt phẳng quỹ đạo gần trùng khít với nhau gọi là \textit{\underline{ mặt phẳng hoàng đạo}}. 4 hành tinh nhỏ vòng trong bao gồm: \textit{Sao Thủy, Sao Kim, Trái Đất và Sao Hỏa}- người ta cũng còn gọi chúng là các \textit{hành tinh đá} do chúng có thành phần chủ yếu từ đá và kim loại. 4 \textit{hành tinh khí khổng lồ} vòng ngoài có khối lượng lớn hơn rất nhiều so với 4 hành tinh vòng trong. 2 hành tinh lớn nhất, \textit{Sao Mộc} và \textit{Sao Thổ} có thành phần chủ yếu từ heli và hidrô; và 2 hành tinh nằm ngoài cùng, \textit{Sao Thiên Vương} và \textit{Sao Hải Vương} có thành phần chính từ băng, như nước, amoniac và mêtan, và đôi khi người ta lại phân loại chúng thành các \textit{hành tinh bằng khổng lồ}.

\textit{Hệ Mặt Trời} cũng chứa 2 vùng tập trung các thiên thể nhỏ hơn. \textit{Vành đai tiểu hành tinh} nằm giữa \textit{Sao Hỏa} và \textit{Sao Mộc}, có thành phần tương tự như các hành tinh đá với đa phần là đá và kim loại. Bên ngoài quỹ đạo của \textit{Sao Hải Vương} là các vật thể ngoài Sao Hải Vương có thành phần chủ yếu từ băng như nước, amoniac và mêtan. Giữa 2 vùng này, có 5 thiên thể điển hình về kích cỡ là \textit{Ceres, Pluto, Haumca, Makemake} và \textit{Eris}, được coi là đủ lớn đủ để có dạng hình cầu dưới ảnh hưởng của chính lực hấp dẫn của chúng, và được các nhà thiên văn phân loại thành \textit{hành tinh lùn}. Ngoài ra có hàng nghìn thiên thể nhỏ nằm giữa 2 vùng này, có kích thước thay đổim như \textit{sao chổi, centaurs} và \textit{bụi liên hành tinh}, chúng di chuyển tự do giữa 2 vùng này.

\textit{Mặt Trời} phát ra các dòng vật chất \textit{plasma}, được gọi là \textit{gió Mặt Trời}, dòng vật chất này tạo ra 1 \textit{bong bóng gió} sao trong \textit{môi trường liên sao} gọi là \textit{\underline{nhật quyển}}, nó mở rộng ra đến tận biên giới của \textit{đĩa phân tán. Đám mây Oort} giả thuyết, được coi là nguồn \textit{sao chổi chu kỳ dài}, có thể tồn tại ở khoảng cách gần 1.000 lần xa hơn nhật quyển.

\end{document}
