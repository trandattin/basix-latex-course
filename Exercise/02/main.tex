\documentclass[twocolumn,12pt]{article}
\usepackage[utf8]{inputenc}
\usepackage[left=3cm, right=2cm,top=2cm, bottom=2cm]{geometry}
%không dùng 2 lệnh geometry và a4paper cùng 1 lúc (chỉ dùng 1 trong 2) 

\usepackage{fancyhdr}
\pagestyle{fancy}
\fancyhf{}
\rhead{{\large{\Large{T}}HE {\Large{B}}LACK {\Large{H}}OLE}}
\lhead{\LaTeXe}
\cfoot{\thepage}
\renewcommand{\headrulewidth}{1pt}
\renewcommand{\footrulewidth}{1pt}

\usepackage{xcolor}

\usepackage{soul} %gói lệnh đề dùng \ul (- là 1 lệnh thay cho \underline)

 \pagecolor{black!10}


\title{\textbf{THE BLACK HOLE}}
\author{Ta Duc Duy }
\date{\today}
% sử dụng make title như này bạn nhé^^ bạn có thể compile thử
%mình out đây, có gì bạn cmt ở bài viết nhé

\renewcommand\thesection{\Roman{section}.} %lệnh thay đổi đánh số thứ tự -> số la mã của mục section và tùy chỉnh đánh số
\renewcommand\thesubsection{\thesection \arabic{subsection}.} % lệnh thay đổi đánh số thứ tự của subsection -> số Ả RẬP và tùy chỉnh đánh số

\begin{document}
%định dạng maketitle ( đưa tiêu đề lên trang bìa )
\begin{titlepage}
	\maketitle
	\pagecolor{black!10} 
\end{titlepage}



\tableofcontents

\section{General}
A \textbf{black hole} is a region of spacetime exhibiting gravitational acceleration so strong that nothing - no particles or even electromagnetic radiation such as light - can escape from it. The theory of general relativity predicts that a sufficiently compact mass can deform spacetime to form a black hole.The boundary of the region from which no es-cape is possible is called the event horizon. Although the event horizon has an enormous effect on the fate and circumstances of an object crossing it, no locally detectable features appear to be observed. In many ways, a black hole acts like an ideal black body, as it reflects no light. Moreover, quantum field theory in curved spacetime predicts that event horizons emit Hawking radiation, with the same spectrum as a black body of a temperature inversely proportional to its mass. This temperature is on the order of billionths of a kelvin for black holes of stellar mass, making it essentially impossible to observe. 

Objects whose gravitational fields are too strong for light to escape were first considered in the 18th century by \textit{John Michell} and \textit{Pierre-Simon Laplace}. The first modern solution of general relativity that would characterize a black hole was found by \textit{Karl Schwarzschild} in 1916, although its interpretation as a region of space from which nothing can escape was first published by \textit{David Finkelstein} in 1958. Black holes were long considered a mathematical curiosity; it was during the 1960s that theoretical work showed they were a generic prediction of general relativity. The discovery of neutron stars by \textit{Jocelyn Bell Burnell} in 1967 sparked interest in gravitationally collapsed compact objects as a possible astrophysical reality.\\ 

Black holes of stellar mass are expected to form when very massive stars collapse at the end of their life cycle. After a black hole has formed, it can continue to grow by absorbing mass from its surroundings. By absorbing
other stars and merging with other black holes, supermassive black holes of millions of solar masses may form. There is consensus that supermassive black holes exist in the
centers of most galaxies. The presence of a black hole can be inferred through its interaction with other matter and with electromagnetic radiation such as visible light. Matter that falls onto a black hole can form an external accretion disk heated by friction, forming some of the brightest objects in the universe. Stars passing too close to a supermassive black hole can be shred into streamers that shine very brightly before being “swallowed”. If there are other stars orbiting a black hole, their orbits can be used to determine the black hole’s mass and location. Such observations can be used to exclude possible alternatives such as neutron stars. In this way, astronomers have identified numerous stellar black hole candidates in binary systems, and established that the radio source known as \textcolor{red}{Sagittarius A*}, at the core of the \textcolor{red}{Milky Way galaxy}, contains a supermassive black hole of about 4.3 million solar masses.\\

On 11 February 2016, \ul{the LIGO collaboration} announced the first direct detection of gravitational waves, which also represented the first observation of a black hole merger. As of December 2018, eleven gravitational wave events have been observed that originated from ten merging black holes (along with one binary neutron star merger). On 10 April 2019, the first ever direct image of a black hole and its vicinity was published, following observations made by \ul{the Event Horizon Telescope} in 2017 of the supermassive black hole in ulMessier 87’s galactic centre.

\section{\textbf{History}}
The idea of a body so massive that even light could not escape was briefly proposed by astronomical pioneer and English clergy-man \textit{John Michell} in a letter published in November 1784. \textit{Michell}’s simplistic calculations assumed such a body might have the same density as \ul{the Sun}, and concluded that such a body would form when a star’s diameter exceeds \ul{the Sun}’s by a factor of 500, and the surface escape velocity exceeds the usual speed of light.\textit{Michell} correctly noted that such supermassive but non-radiating bodies might be detectable through their gravitational effects on nearby visible bodies. Scholars of the time were initially excited by the proposal that giant but invisible stars might be hiding in plain view, but enthusiasm dampened when the wavelike nature of light became apparent in the early nineteenth century.

If light were a wave rather than a “corpuscle”, it is unclear what, if any, influence gravity would have on escaping light waves. Modern physics discredits\textit{ Michell}’s notion of a light ray shooting directly from the surface of a supermassive star, being slowed down by the star’s gravity, stopping, and then free-falling back to the star’s surface.

\subsection{\textbf{General relativity}}
In 1915, \textit{Albert Einstein} developed his theory of general relativity, having earlier shown that gravity does influence light’s motion. Only a few months later, \textit{Karl Schwarzschild} found a solution to the Einstein field equations, which describes the gravitational field of a point mass and a spherical mass. A few months after \textit{Schwarzschild, Johannes Droste}, a student of\textit{ Hendrik Lorentz}, independently gave the same solution for the point mass and wrote more extensively about its properties. This solution had a peculiar behaviour at what is now called the Schwarzschild radius, where it became singular, meaning that some of the terms in the Einstein equations became infinite. The nature of this surface was not quite understood at the time. In 1924, \textit{Arthur Eddington} showed that the singularity disappeared after a change of coordinates, although it took until 1933 for \textit{Georges  Lema\^{i}tre} to realize that this meant the singularity at the Schwarzschild radius was a non-physical coordinate singularity. \textit{Arthur Eddington} did however comment on the possibility of a star with mass compressed to the Schwarzschild radius in a 1926 book, noting that \textit{ Einstein}’s theory allows us to rule out overly large densities for visible stars like Betelgeuse because \textbf{“A star of 250 million km radius could not possibly have so high a density as the sun. Firstly,the force of gravitation would be so great that light would be unable to escape from it, the rays falling back to the star like a stone to the earth. Secondly, the red shift of the spectral lines would be so great that the spectrum would be shifted out of existence. Thirdly, the mass would produce so much curvature of the space-time metric that space would close up around the star, leaving us outside (i.e., nowhere).”}\\ 

In 1931, \textit{Subrahmanyan Chandrasekhar} calculated, using special relativity, that a non-rotating body of electron-degenerate matter above a certain limiting mass (now called the Chandrasekhar limit at 1.4 M) has no stable solutions. His arguments were opposed by many of his contemporaries like \textit{Eddington} and \textit{Lev Landau}, who argued that some yet unknown mechanism would stop the collapse. They were partly correct: awhite dwarf slightly more massive than the Chandrasekhar limit will collapse into a neutron star, which is itself stable. But in 1939, \textit{Robert Oppenheimer} and others predicted that neutron stars above another limit (the Tolman-Oppenheimer-Volkoff limit) would collapse further for the reasons presented by Chandrasekhar, and concluded that no law of physics was likely to intervene and stop at least some stars from collapsing to black holes. Their original calculations, based on the Pauli exclusion principle, gave it as 0.7 M; subsequent consideration of strong force-mediated neutron-neutron repulsion raised the estimate to approximately 1.5 M to 3.0 M. Observations of the neutron star merger \ul{W170817}, which is thought to have generated a black hole shortly afterward, have refined the TOV limit estimate to 2.17 M.

Oppenheimer and his co-authors interpreted the singularity at the boundary of the Schwarzschild radius as indicating that this was the boundary of a bubble in which time stopped. This is a valid point of view for external observers, but not for infalling observers. Because of this property, the collapsed stars were called “frozen stars”, because an outside observer would see the surface of the star frozen in time at the instant where its collapse takes it to the Schwarzschild radius.

\section*{\textbf{Reference}} 
\addcontentsline{toc}{section}{Reference} %lệnh đưa phần \section*{\textbf{Reference}} lên mục lục
\hspace{1.5cm} “Black hole” - Wikipedia
\end{document}
