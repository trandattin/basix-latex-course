\documentclass[12pt,twoside]{article}
\usepackage[utf8]{vietnam}
\usepackage[left=3.5cm, right=2cm,
top=2cm, bottom=2cm]{geometry}
% Gói văn bản mẫu
\usepackage{blindtext}
% Gói lệnh màu
\usepackage{xcolor}
% Gói lùi đầu dòng
\usepackage{indentfirst}
% Thiết lập tất cả các đoạn lùi 5cm
\setlength{\parindent}{5cm}

% Gói điều chỉnh footer and header
\usepackage{fancyhdr}
\pagestyle{fancy}
\fancyhf{}
\rhead{\LaTeX}
\lhead{Định dạng văn bản}
\cfoot {Trang \thepage}
\fancyhead[LE,RO]{\LaTeX}
\fancyhead[RE,LO]{Định dạng văn bản}
\fancyfoot[LE,RO]{\thepage}
\renewcommand{\headrulewidth}{3pt}
\renewcommand{\footrulewidth}{1pt}

% Tạo Wartermark và các thông số
\usepackage{draftwatermark}
\SetWatermarkText{\LaTeX\ Basics}
\SetWatermarkScale{0.5}
\SetWatermarkAngle{0}
\SetWatermarkColor{blue!50}
%\SetWatermarkLightness{0.5}
%\SetWatermarkHorCenter{3cm}
%\SetWatermarkVerCenter{3cm}

% Tạo title
\title{Định dạng văn bản}
\author{Quỳnh Trang}
\date{\today}

% Đánh dấu la mã cho section
\renewcommand\thesection{\roman{section}}

% Đổ màu nền
%\pagecolor{yellow!20!}
% Đổ màu chữ
%\color{blue}

\newcommand{\prmt}[1]{\textit{\fontfamily{cmr}
\selectfont$\langle$#1$\rangle$}}
\newenvironment{Example}[2][Example]
{This is an #1. You gave #2 as an argument.
The rest will be bold: \bfseries}
{}

% Document
\begin{document}
% Gọi title
\maketitle
% Gọi mục lục
\tableofcontents

\section*{Lời mở đầu}
\addcontentsline{toc}{section}{Lời mở đầu}

\section{Mục 1} 
\blindtext

\newpage
\subsection{Tiểu mục}
\blindtext

\subsubsection{Tiểu mục con}
\blindtext

\section{Lại là một mục nữa =))}
\blindtext

% Tạo trang không footer and header
%\thispagestyle{empty}

% Căn chỉnh đoạn văn
%\centering Xin chào mọi người
%\flushright Hôm nay là chủ nhật
%\flushleft Mọi người đi học đầy đủ không?

% Khoảng cách chữ
\noindent a \quad b\\
a \qquad b\\
a \hspace{2cm} b\\
a \hfil b \hfil c\\
a \hfill b \hfill c

\newpage

\blindtext
% Khoảng cách đoạn
\vfill
\blindtext

\newpage

% Đổ màu trang
\pagecolor{yellow!30!}
% Đổ màu chữ
\textcolor{blue}{ĐH Bách Khoa HN}
% Đổ màu nền chữ
\colorbox{red}{Viện Toán ứng dụng và Tin học}\\


% Định nghĩa màu
\definecolor{HustRed}{RGB}{206,22,40}

% Dùng newenvironment
\begin{Example}{Theorem}
	Đây là một định lý. 
\end{Example}

\end{document}