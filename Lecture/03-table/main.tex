\documentclass[12pt]{article}
\usepackage[utf8]{vietnam}

% Điều chỉnh ký hiệu mặc định của gói
\usepackage{enumerate}

% Không sử dựng chung với gói enumerate
% Gói enumitem dùng để điều chỉnh khoảng cách giữa các \item
%\usepackage{enumitem}
\renewcommand{\labelenumii}{\Roman{enumii}}
\renewcommand{\labelenumi}{$\star$}

% Môi trường tham chiếu
\usepackage{paralist}

\usepackage{tasks}

%\usepackage{array}

\usepackage{tabu}

\begin{document}

\begin{enumerate}
	\item One
	\begin{enumerate}
		\item ex 1
		\item ex 2
		\item ex 3
	\end{enumerate}
		\item Two
\end{enumerate}

\begin{description}
	\item[Hàm số] là trường hợp đặc biệt của ánh xạ
\end{description}
	
Điều cần phải làm
\begin{asparaenum}[I]
	\item điều 1
	\item điều 2 \label{n1}
\end{asparaenum}
	
Nhưng không cần làm
\begin{inparaenum}[(a)]
	\item điều 3
	\item điều 4 \label{n2}
\end{inparaenum}\\

Hôm nay hãy làm \ref{n1} nhưng không làm \ref{n2}.

\begin{tasks}[label=\roman*. ](4) %set số task mỗi dòng
	\task câu 1
	% Dấu * để xuống dòng
	\task* câu 2
	\task câu 3
	\task câu 4
	\task câu 5
	\task câu 6
	\task câu 7
	\task câu 8
	\task câu 9
\end{tasks}

\newpage

% Các float để đặt vị trí tabular bằng môi trường table
% h là here, b là bottom, t là top, p là special page và ! là ghi đè mặc định
\begin{table}[h]
\begin{center}
Một bảng đơn giản 
\begin{tabular}{ | p{2cm} | m{2cm} | m{2cm} |}
	\hline
	ô 1  & ô 2 & ô 3\\ [0.5ex] 
	\hline\hline
	ô 3 & ô 4 & ô 5\\
	\cline{1-2} 
	ô 6 & ô 7 & ô 8\\
	\hline
\end{tabular}
\end{center}
	\caption{Bảng}
	\label{b1}
\end{table}
Gọi tham chiếu: Bảng \ref{b1}

\newpage
% Gói tabu
\begin{tabu} to 0.8\textwidth {|X[c]|X[c]|X[c]|}
	\hline
	item 11 & item 12 & item 13 \\
	\hline
	item 21 & item 22 & item 23 \\
	\hline
\end{tabu}

\end{document}