\documentclass[12pt]{article}
\usepackage[utf8]{vietnam}
% Gói chèn cột
\usepackage{multicol}
% Gói chèn hàng
\usepackage{multirow}
% Gói tô màu nền cho table
\usepackage[table]{xcolor}
% Điều chỉnh độ dày viền
\setlength{\arrayrulewidth}{1.5pt}
% Điều chỉnh độ rộng hai bên trong mỗi ô trong bảng là 1 cm
\setlength{\tabcolsep}{1cm}
% Điều chỉnh khoảng cách phía trên và dưới trong bảng rộng {3} lần so với bình thường
\renewcommand{\arraystretch}{3}

\newcolumntype{f}{>{columncolor{red!10}}haligni}

\begin{document}

% Tạo màu cho viền của bảng
\arrayrulecolor{red}
% Tô màu cho các ô 
% \rowcolors{tô màu từ dòng=}{ô lẻ}{ô chẵn}
{\rowcolors{1}{green!10}{blue!10}
\begin{tabular}{|c|c|c|}
	\hline
	% Tô màu cho dòng riêng biệt
	% \rowcolor{màu!độ nhạt}
	% Ghép cột
	% \multicolumn{số ô}{cú pháp}{nội dung}
	\rowcolor{yellow!20}\multicolumn{2}{|c|}{ô 1 và 2} & ô 3 \\
	% Ghép hàng
	% \multirow{số ô}{khoảng cách}{nội dung}
	\multirow{2}{1.5cm}{ô 4 và 7} & ô 5 & ô 6 \\
	& ô 8 & ô 9 \\
	\hline
\end{tabular}}

\vspace{3cm}

\begin{tabular}{|c|c|c|}
	\hline
	\multicolumn{2}{|c|}{\cellcolor{red!10}ô 1 và 2} & \cellcolor{blue!10}ô 3 \\
	\hline
	\multirow{2}{1.5cm}{ô 4 và 7} & ô 5 & ô 6 \\
	& ô 8 & ô 9 \\
	\hline
\end{tabular}

\begin{center}
	\begin{tabular}{| p{2cm} | p{1cm} | m{2cm} | p{1cm} |}
	\hline
	Col1 & Col2 & Col2 & Col3 \\ [0.5ex] \hline
	\hline
	1 \newline dòng thứ 2 & 6 & 87837 & 787 \\
	\hline
	\end{tabular}
\end{center}

\end{document}